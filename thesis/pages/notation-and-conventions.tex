\chapter*{Notation and conventions}
\phantomsection\addcontentsline{toc}{chapter}{\protect Notation and conventions}
\thispagestyle{empty}

Throughout this thesis, we choose the signature for the metric of spacetime to be \\
${\eta_{\mu\nu} = \diag(-,+,+,+)}$.
This is a list of books on astrophysics, cosmology and general relativity I have came across while working on this thesis, divided into the different sign convention for the metric of spacetime. 
\begin{itemize}
    \item[$(\bs{+})$] $\eta_{\mu \nu} = \diag(-,+,+,+)$
    \begin{itemize}
        \item[$\bullet$] Matthias Bartelmann. \textit{Das kosmologische Standardmodell} \cite{Bartelmann2019}
        \item[$\bullet$] Sean M. Carroll. \textit{Spacetime and Geometry} \cite{SeanCarroll2019}
        \item[$\bullet$] Scott Dodelson, Fabian Schmidt. \textit{Modern Cosmology} \cite{Dodelson2020}
        \item[$\bullet$] C.W. Misner, K.S. Thorne, J.A. Wheeler. \textit{Gravitation} \cite{MTW2017}
        \item[$\bullet$] Bernard Schutz. \textit{A First Course in General Relativity} \cite{Schutz2009}
        \item[$\bullet$] Karl-Heinz Spatscheck. \textit{Astrophysik} \cite{Spatschek2017}
        \item[$\bullet$] Steven Weinberg. \textit{Cosmology} \cite{Weinberg2008}
        \item[$\bullet$] A. Zee. \textit{Einstein Gravity in a Nutshell} \cite{Zee2013}
    \end{itemize}
    \item[$(\bs{-})$] $\eta_{\mu \nu} = \diag(+,-,-,-)$
    \begin{itemize}
        \item[$\bullet$] Bradley W. Carroll, Dale A. Ostlie. \textit{An Introduction to Modern Astrophysics} \cite{BradleyCarroll2007}
        \item[$\bullet$] Viatcheslav Mukhanov. \textit{Physical Foundations of Cosmology} \cite{Mukhanov2005}
        \item[$\bullet$] P.J.E. Peebles. \textit{Principles of Physical Cosmology} \cite{Peebles1993}
    \end{itemize}
\end{itemize}

\noindent In order not to lose the connection between empirically measured quantities we can observe and theoretical quantities introduced by the cosmological models, we choose to let $c = \SI{299 792 458}{\meter/\second}$ and $G = \SI{6.674e-11}{\meter^3/(\kilogram \ \second^2)}$ in SI-units.\footnote{... although the convenience for mathematical manipulations in the case of natural units where $c = G = 1$ cannot be denied.}
