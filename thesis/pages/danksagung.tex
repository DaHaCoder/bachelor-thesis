\chapter*{Danksagung}
\phantomsection\addcontentsline{toc}{chapter}{\protect Danksagung}

\noindent Es war ein langer Weg, bis ich endlich diesen Punkt in meinem Leben erreicht habe. Es war nicht einfach. Aber niemand sagte, dass das Leben einfach wird.

\noindent Ich war etwa 12 Jahre alt, als ich das erste Mal mit meinem kleinen Teleskop Ausschau nach dem schönsten Objekt unseres Sonnensystem hielt. In der Lage zu sein, die Ringstruktur des Saturn zu bewundern, öffnete mir die Augen und das Herz zur Erkundung der Schönheit der Natur - der Schönheit des Weltraums.

Während wir aufwachsen, versuchen wir - getrieben von Neugier und Faszination - die Welt um uns herum zu verstehen. Wir sind auf der Suche nach Antworten auf die ``großen Fragen des Lebens''.

\noindent Woher kommen wir ? \\
Warum ist das Universum entstanden ? \\
Wie alt ist das Universum ? \\
Welche Mechanismen bestimmen die Geschicke der Welt ? \\
Was wird in der Zukunft passieren ? \\
Welche Rolle spiele ich in diesem unvorstellbar großen Universum ? \\
Welchen Sinn hat mein Leben ? \\
Und, wohin möchte ich gehen ? 

\noindent Das sind gewiss keine einfachen Fragen. Und das ist wahrscheinlich auch der Grund, warum dies die ältesten Fragen der Menschheit sind. \\
Im Laufe meiner ``Reise durch das Leben'' versuchte ich, Antworten auf diese Fragen zu finden. Vielleicht schaffe ich es, auf einige dieser Fragen mögliche Antworten in dieser Arbeit geben zu können. Vielleicht werde ich auf einige dieser Fragen nie eine Antwort finden.
Aber zumindest auf die Frage nach dem Sinn \textit{meines} Lebens habe ich eine Antwort gefunden: die Schönheit der Natur zu entdecken. Und während dieses Abenteuers niemals aufzugeben - egal, wie hart und frustrierend das Leben manchmal sein kann. Egal, wie oft ich gefallen bin oder noch auf den harten Boden der Realität fallen werde. Denn es ist meine Neugier, meine Ambition und meine Leidenschaft, welche mich dazu bringt, wieder aufzustehen und meine Augen, meinen Blick in den Himmel zu richten.

\noindent Das ist wohl die wichtigste Lektion, welche ich in den letzten Jahren im Laufe des Physikstudiums gelernt habe.

\noindent Die Tatsache, es so weit gebracht zu haben, ist jedoch nicht nur dem Lesen von Lehrbüchern, dem Rechnen von Übungsblättern oder dem Vorbereiten auf Klausuren geschuldet. Denn es ist nicht nur der Fleiß, den ich erbracht habe.
Ich stehe, wo ich bin, dank dem Fleiß bestimmter Menschen, die in mein Leben kamen. Ich bin äußert dankbar, dass ich das Glück hatte, jene Menschen zu begegnen. Und nun ist wohl der beste Anlass, um ihnen meinen herzlichsten Dank mitzuteilen.

\noindent In erster Linie möchte ich meinen Lehrern danken, welche mir nicht nur viel beibrachten, sondern auch ihr Bestes gaben, um auf meine schwierigen Fragen Antworten zu finden und meine Wissensbegierde zu stillen, was wahrscheinlich nicht immer einfach war und viel Geduld erforderte,


