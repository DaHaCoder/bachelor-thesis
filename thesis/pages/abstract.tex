\chapter*{Abstract}
\phantomsection\addcontentsline{toc}{chapter}{\protect Abstract}
\thispagestyle{empty}

In 2011, Saul Perlmutter, Adam G. Riess and Brian P. Schmidt received the Nobel Prize in Physics \enquote{for the discovery of the accelerating expansion of the Universe through observations of distant supernovae}. \footnote{\href{https://www.nobelprize.org/prizes/physics/2011/press-release/}{Press release: https://www.nobelprize.org/prizes/physics/2011/press-release/}} \\
Since then, we know that the expansion of the universe is accelerating. 
The cause of this accelerated expansion is still unknown, but there are some theoretical models which attempt to explain this phenomenon. \\

\noindent The established model in recent cosmology is the \textbf{$\Lambda$CDM-Model}, sometimes referred as the \textit{standard model of cosmology}. 
In this model, the cause of the accelerated expansion is due to the so called \textit{cosmological constant} $\Lambda$, which appears as a physical constant in Einstein's field equations of general relativity like the Newtonian gravitational constant $G$. \\

\noindent Another model, published in April 2000 by Gia Dvali, Gregory Gabadadze and Massimo Porrati, -- the \textbf{DGP-Model} -- proposes a modification of Einstein's field equations by introducing a fifth dimension to the four-dimensional spacetime, so that gravity behaves equivalently to Newtonian gravity on small distances, but weakens on large scales. \\

\noindent In this thesis, we consider measurements of type Ia Supernovae by the \textit{Supernova Cosmology Project} \footnote{\href{https://supernova.lbl.gov/Union/}{Supernova Cosmology Project: https://supernova.lbl.gov/Union/}} (dataset ``Union2.1'') to obtain best-fit values to the free parameters and constraints to both cosmological models.
