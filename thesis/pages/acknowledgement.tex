\chapter*{Acknowledgement}
\phantomsection\addcontentsline{toc}{chapter}{\protect Acknowledgement}
\thispagestyle{empty}

\epigraph{\textit{``The effort to understand the universe is one of the very few things which lifts human life a little above the level of farce and gives it some of the grace of tragedy.''}}{Steven Weinberg, \cite[Epilogue]{Weinberg1993}}

\noindent It was a long path to finally achieve this point in my life. And it was not an easy one. But no one said that life was going to be easy.

\noindent I was about 12 years old when I watched the sky the very first time with my telescope, looking out for the most beautiful planet of our solar system. Being able to admire the ring structure of Saturn opened my eyes and my heart to explore the beauty of nature, the beauty of space. \\

\noindent While growing up, we try, driven by curiosity and fascination, to understand the world around us. We may seek answers to ``the big questions of life''. \\ 

{ \itshape
\noindent Where do we come from? \\
Why did the universe come to existence? \\
How old is the universe? \\
Which mechanisms in nature guide the course of all events? \\
What will happen in the future? \\
Which role do I play in this unimaginable big universe? \\
What is the purpose of my life? \\
And, where do I want to go? \\
}

\noindent Those are difficult questions. And that is why those questions are probably the oldest ones of humanity. \\
Throughout my ``journey of life'', I tried to find answers to those questions. Maybe I could find some attempts to answer a few of those in this thesis. Maybe I will never find an answer to some of these questions. \\

\noindent At least I found an answer what the purpose of \textit{my} life is: to discover the beauty of nature. Going on with my journey, trying to find answers. And never giving up during this adventure -- no matter, how hard and how frustrating life can be sometimes. No matter how often I have fallen or I will fall down to the harsh ground of reality. Because it is my curiosity, my ambition and my passion that makes me standing up again to turn my eyes, my gaze into the sky. \\ 

\noindent That is the most important lesson I have learned in the last years, studying physics. \\ 

\noindent The fact that I made it so far is not only through reading textbooks, doing my exercise sheets or preparing for exams. Because it is not only the effort I brought up. \\
I am where I am thanks to the effort of several people that came into my life. I am very glad that I had the luck to get to know these people. And therefore, this is the best occasion for me to thank them. \\

\noindent First of all, I want to thank my teachers, who not only taught me a lot, but also did their best to answer my difficult questions and quench my thirst for knowledge, which was probably not always easy and took a lot of patience, \\

\noindent {\bfseries Pierette Al-Korey, Ulrike Blattert, Peter Bodden, Nicole Bömecke, Jörg Bühler, Reiner Büter, Rolf Heckmann, Petra Jähnigen, Thomas Schröder Klementa, Axel Knuth, Eva Krüger, René Meyer-Brede, Angelika Nieboer, Roland Petereit, David Stephan, Volkhard Stierhof, Stefan Usée, Selma Weiß-Tümmers, Wolf Wingenfeld, Sema Yilmaz, Mike Ziegner.} \\

\noindent Further, I want to thank my supervisors \textbf{Jochen} and \textbf{Steffen}, who offered me this opportunity to write my thesis on cosmology and kindly assisted me at any time with my questions. \\

\noindent I want to thank \textbf{Silas}, who provided me his bachelor thesis on a similar topic, with which I was able to compare the results I obtained during my analysis and evaluations. \\

\noindent I would like to thank all my colleagues, fellow students and friends whom I have met and who accompanied me on my way through the bachelor studies and always stood by me -- especially to \textbf{Arthur} and \textbf{Marc}, who can bring a smile to my face even on the saddest days. \\

\noindent I want to thank \textbf{Yannick}, not only for his advice on how to reduce the computation time of my codes from several hours to a few seconds and proofreading this thesis, but for being a good friend. \\

\noindent My heartfelt thanks goes to \textbf{Simon} -- the most kind-hearted person I have ever met. By submitting this bachelor thesis, I would like to fulfill my promise to him to never give up my bachelor studies and successfully achieve the bachelor's degree in physics. \textbf{Thank you Simon}. \\

\noindent And last but not least, I would like to express my greatest gratitude to my family -- especially \textbf{my father}, \textbf{my mother} and \textbf{my sister}, who raised me and cared for me since the day I was born, who provided me everything that was needed to enjoy an excellent education, who opened the path on my way to a self-determined life and contributed to becoming the person I am today. \\

\noindent From the bottom of my heart -- thank you.
