\chapter{Conclusion}
\thispagestyle{empty}

\noindent First of all, we see that the computational implementation carried out in this thesis are able to reproduce the results of the \textit{Supernovae Cosmology Project} using the ``Union2.1'' SN Ia dataset.
Further, the result of $\Omega_{\text{m},0,\text{best}} = 0.28$ in both cases, $\vb*{\theta} = (\Omega_{\text{m},0}, \Omega_{\Lambda,0})$ and $\vb*{\theta} = (\Omega_{\text{m},0}, \alpha)$, shows consistency in the computational implementation of our parameter estimation. \\

\noindent The obtained values of $(\Omega_{\text{m}, 0, \text{best}}, \Omega_{\Lambda, 0, \text{best}}) \pm (\sigma_{\text{m}, 0}, \sigma_{\Lambda,0}) = (0.28, 0.72) \pm (0.07, 0.12)$ and \\
${(\Omega_{\text{m}, 0, \text{best}}, \alpha_{\text{best}}) = (0.28, -0.05)}$ prefer the $\Lambda$CDM-model for a flat universe. Yet, the large errors, especially in the case of analyzing the parameter pair $\vb*{\theta} = (\Omega_{\text{m},0}, \alpha)$, do not allow a final judgement on the cosmological models by using the ``Union2.1'' SN Ia dataset alone (as mentioned in \cite{Thomas2009}). This can also be seen in Figure \ref{fig:redshift-vs-distance-modulus}, where the predicted relation between distance modulus and redshift in both the $\Lambda$CDM-model and the DGP-model match even for high-redshift datapoints and therefore do not allow a clear distinction. Only under consideration of multiple measurements as the cosmic microwave background (CMB), baryonic acoustic oscillations (BAO) and weak gravitational lensing, it is possible to provide constraints that favour the $\Lambda$CDM-model and disfavour the DGP-model. \\

\noindent Under the assumptions of homogeneity and isotropy on large scales, correct description of gravity by general relativity and supernovae type Ia as standard candles, we can conclude that the $\Lambda$CDM-model is at the current state of research in physical cosmology one of the best theoretical models we can provide that matches with multiple data and observations. \\

\noindent Nevertheless, the physical interpretation of the cosmological constant $\Lambda$ as corresponding to vacuum energy density is questionable, since the obtained value for vacuum energy density by cosmological observations differ highly from predictions made by quantum field theory. This problem is known as the ``cosmological constant problem'' (see \cite{Weinberg1989}). As long as this problem is not considered to be solved, there is at least one legitimate concern for developing cosmological models that refrain from introducing a cosmological constant as provided by the DGP-model. \\
Furthermore it must be ensured that supernovae of type Ia can be seen as good candidates for standard candles. \\

\noindent Finally, we can conclude that cosmological parameter estimation should always account for multiple datasets, observations and methods to provide constraints and therefore good predictions on which cosmological model is most suitable for describing the universe we observe. 
