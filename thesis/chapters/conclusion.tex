\chapter{Conclusion}
\thispagestyle{empty}

\noindent First of all, we see that the computational implementation carried out in this thesis are able to reproduce the results of the \textit{Supernovae Cosmology Project} using the ``Union2.1'' SN Ia dataset.
Further, the result of $\Omega_{\text{m},0,\text{best}} = 0.28$ in both cases, $\vb*{\theta} = (\Omega_{\text{m},0}, \Omega_{\Lambda,0})$ and $\vb*{\theta} = (\Omega_{\text{m},0}, \alpha)$, shows consistency in the computational implementation of our parameter estimation. \\

\noindent The obtained values of $(\Omega_{\text{m}, 0, \text{best}}, \Omega_{\Lambda, 0, \text{best}}) \pm (\sigma_{\text{m}, 0}, \sigma_{\Lambda,0}) = (0.28, 0.72) \pm (0.07, 0.12)$ and \\
${(\Omega_{\text{m}, 0, \text{best}}, \alpha_{\text{best}}) = (0.28, -0.05)}$ prefer the $\Lambda$CDM-model for a flat universe. Yet, the large errors, especially in the case of analyzing the parameter pair $\vb*{\theta} = (\Omega_{\text{m},0}, \alpha)$, do not allow a final judgement on the cosmological models by using the ``Union2.1'' SN Ia dataset alone (as mentioned in \cite{Thomas2009}). This can also be seen in Figure \ref{fig:distance-modulus-vs-redshift}, where the predicted relation between distance modulus and redshift in both the $\Lambda$CDM-model and the DGP-model match even for high-redshift datapoints and therefore do not allow a clear distinction. Only under consideration of multiple measurements as the cosmic microwave background (CMB), baryonic acoustic oscillations (BAO) and weak gravitational lensing, it is possible to provide constraints that favour the $\Lambda$CDM-model and disfavour the DGP-model.

\noindent Nevertheless, the physical interpretation of the cosmological constant $\Lambda$ needs 
